\documentclass[twocolumn]{revtex4}
\usepackage{graphics,graphicx,epsfig,amsmath,multirow,gensymb,commath,textcomp,natbib,blindtext,mhchem,tabularx,array,makecell,threeparttable,amssymb,relsize,csquotes}
\usepackage{listings}
\usepackage[a4paper, left=1.85cm, right=1.85cm, top=1.85cm, bottom=1.85cm]{geometry}   
\usepackage[normalem]{ulem}
\newcommand{\squeezeup}{\vspace{-2.5mm}}

\def\bibsection{\section*{\refname}} 
\renewcommand{\thesubsection}{\alph{subsection}}

\renewcommand\theadalign{bc}
\renewcommand\theadfont{\bfseries}
\renewcommand\theadgape{\Gape[4pt]}
\renewcommand\cellgape{\Gape[4pt]}

\begin{document}

\textheight=26.385cm
%Change textheight as the last resort...

\title{Constraining the geometry of the Universe using Type Ia supernovae}
 
\author{Jacky Cao, Group C3, Physics Problem Solving \\ Date of report: 28/02/2018}

\begin{abstract}              
Type Ia supernovae have the unique trait of being standard candles, their light curves can be used in cosmology to calculate and constrain cosmological parameters. In observing Type Ia supernovae and fitting model light curves to such data one can attempt to derive such values. We have monitored, collected, and analysed data for supernova explosions over a period of 34 days. A $16''$ and a $0.5$ m telescope situated in Durham and La Palma was used for this project. After calculating the magnitudes for a Type Ia (2017hhz) and Type Ia-91bg (2017hle) supernova object, we fitted template light curves with the Python program, \textit{SNooPy}. The quality of fit for the program's \texttt{fit()} function was deemed to be acceptable in accordance to the average reduced $\chi^2$ values calculated for the B and V photometric bands, $\chi^2_{\nu,B} \approx 1.38$ and $\chi^2_{V,\nu} \approx 2.95$ - a good fit requiring $\chi^2_{\nu} \approx1$. The distance modulus to the supernova 2017hhz was calculated by SNooPy to be $36.121\pm0.106$ mag, using this value we were able to compute $H_0=70\pm20$ km s$^{-1}$ Mpc$^{-1}$. However, the quoted error negates the meaning of $H_0$ as it is too large of an uncertainty. In improving the accuracy and uncertainties we suggest that more observations of the supernovae were required, and constraining values should be used for the parameters in SNooPy's templates.
\end{abstract}

\maketitle

\vspace{-3ex}
\section{Introduction and Theory} 
\vspace{-2ex}
In cosmology, one can argue that one of the most important observable events are supernova explosions. As a massive main sequence star runs out of nuclear fuel to burn, the equilibrium configurations which initially provided structure will cease to exist. What follows is a cataclysmic supernova explosion \cite{longair}. 

We can generally class supernova explosions into two separate groups, Type I and Type II supernovae. The main distinction arises due to the fact that Type I's have an optical spectra which contains no Balmer hydrogen features, whilst Type II supernovae do contain this hydrogen feature \cite{mod_ast}. 

Within these two subclasses there are further divisions which can be characterised through their spectra and through features as found in their light curves \cite{obs_phys_class_sn}. Light curves are a way to show the evolution of a supernova's magnitude as time passes. With two of the subclasses of Type II supernovae, Type II-L and Type II-P, they have been classed by features of their respective light curves: a rapid linear decrease in magnitude and constant magnitude \cite{mod_ast}.

For the application of supernovae in cosmology, we must turn to supernovae of the Type Ia variety and to the light curves that they produce. 

\vspace{-3ex}
\subsection{Type Ia Supernovae}
\vspace{-2ex}
It is generally accepted that the light curves of Type Ia supernovae are homogeneous \cite{posn}, this means that they can be utilised as standard candles, therefore allowing us another measure of cosmic distance. 

Their homogeneity arises due to the mechanism behind their explosion. The progenitor system for Type Ia's consist of a binary system with a white dwarf and another star which has a mass close towards the Chandrasekhar limit of $1.4 M_{\odot}$ \cite{mod_ast, posn}. The white dwarf accretes matter from it's companion until it itself reaches this critical mass limit. After this point we would then expect the primary white dwarf to collapse into a neutron star, however this is not the case. Instead, we find that a supernova explosion occurs due to a disruption to it's internal structure.

It is currently understood that as the primary white dwarf star accretes matter it heats up and produces thermonuclear energy. This energy which is achieved in the stellar interior at a temperature of $10^9$ K disrupts the electron degeneracy pressure. The star becomes gravitationally unstable due to the disruption from the nuclear energy, a total collapse into a neutron star is thus prevented \cite{longair, posn}.

This particular mechanism can thus be used to attribute for the standard profile of Type Ia light curves. Using these graphs, the maximum B photometric band magnitude can be obtained from the following equation,
\begin{equation}
M_{\max}(B)=-21.726+2.698\Delta m_{15}(B),
\end{equation}

where $\Delta m_{15}(B)$ is the decline rate parameter or Phillips' parameter \cite{high_en_astro}. This is defined as the mean rate of decline of the B band light curve from peak light to 15 days after the maximum had been achieved \cite{abs_phil}. This parameter relates the apparent magnitude to time and provides a way to compare and contrast different Type Ia supernovae. 

With values for the absolute magnitude of Type Ia supernovae and with obtained values of redshift from spectroscopic observations of them, the distance to them can then be calculated using the following equation, 
\begin{equation}
\mu = 5 \log_{10}(d) - 5,
\end{equation}

where $\mu$ is the distance modulus and can be defined as the apparent magnitude minus the absolute magnitude of supernova, and $d$ is the distance to the supernova in parsecs \cite{mod_ast}. 

Within cosmology Type Ia supernovae can thus be employed as standard candles to measure cosmic distances, however they can also be used to calculate the geometry of the universe.

\vspace{-3ex}
\subsection{Cosmological Parameters}
\vspace{-2ex}
At large cosmological distances, the appearance of objects will be affected by the spacetime which it travels through. If we therefore wanted to describe the geometrical properties of the universe, we would require to solve Einstein's general theory of relativity.

If we solve the field equations for an isotropic, homogenous universe a description of the evolution of the universe can be obtained in the form of a differential equation, the Friedmann equation \cite{mod_ast}. In the 1922 form of the equation, a nonstatic universe is accounted for,
\begin{equation}
\Big[ \Big( \frac{1}{R} \frac{dR}{dt} \Big)^2 - \frac{8}{3} \pi G \rho \Big] R^2 = -k c^2,
\end{equation}

where $R(t)$ is the scale factor, $G$ the gravitational constant, $\rho$ is the mass density, and $k$ a parameter which describes the shape of the universe. 

Work performed by Einstein led to a cosmological constant $\Lambda$ being introduced in the Friedmann equation. This was added in the form of $\tfrac{1}{3}\Lambda c^2$ within the left-hand bracketed section, a result arising from Newtonian cosmology \cite{mod_ast}.

Whilst Einstein included this term to account for his static and closed universe, astronomers in the 1990s eventually related the cosmological constant to dark energy as a consequence of observations of an accelerating universe \cite{mod_ast}. 

\vspace{-3ex}
\subsection{Cosmological Parameters from Type Ia Supernovae}
\vspace{-2ex}
These parameters can thus be calculated using 

\vspace{-3ex}
\subsection{Project Aims}
\vspace{-2ex}

Through computational analysis it is possible to employ data sets of Type Ia supernovae to constrain values for cosmological parameters. In Section \ref{sec:results_discussion} we discuss the results from our entire experimental period, from our early work with Least-Square statistics to later experimentation with the Markov-Chain-Monte-Carlo method. 

\vspace{-3ex}
\section{Results and Discussion} 
\label{sec:results_discussion}
\vspace{-2ex}
From our experimentation we have therefore found results which ??

\vspace{-3ex}
\subsection{$\chi^2$ Statistics} 
\vspace{-2ex}

asdasd

\vspace{-3ex}
\subsection{Bayesian Statistics} 
\vspace{-2ex}

asdasd

\vspace{-3ex}
\subsection{Further Investigation} 
\vspace{-2ex}

asdasd

\vspace{-5ex}
\section{Conclusions}
\vspace{-2ex}

In conclusion, we have found that it is possible to constrain cosmological parameters of the universe using Type Ia supernova data.

\vspace{-3ex}
\bibliographystyle{abbrv}
\bibliography{supernova_cosmology}

\clearpage
\appendix

\vfill
\twocolumngrid
\vspace{-3ex}
\section{??}
\vspace{-2ex}



\clearpage
\end{document}