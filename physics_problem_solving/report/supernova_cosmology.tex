\documentclass[twocolumn]{revtex4}
\usepackage{graphics,graphicx,epsfig,ulem,amsmath,multirow,gensymb,commath,textcomp,natbib,blindtext}
\newcommand{\squeezeup}{\vspace{-2.5mm}}

\begin{document}

\textheight=26.385cm
%Change textheight as the last resort...

\title{Supernova Cosmology}
 
\author{Jacky Cao, Physics Problem Solving \\ Date of report: 19/03/2017}

\begin{abstract}              
asdsa
\end{abstract}

\maketitle

\vspace{-3ex}
\section{Introduction} 
\vspace{-2ex}
\subsection{Supernovae}
\vspace{-2ex}

In cosmology, we can argue that one of the most important objects that we can observe are supernovae.  

\vspace{-3ex}
\subsection{Supernova Discovery}
\vspace{-2ex}

asdasd

\vspace{-3ex}
\subsection{Project Aims}
\vspace{-2ex}

Studying the evolution of the magnitudes of different supernovae will allow us to produce a light curve which can be fitted with known models. Through this we can then discover the type of supernova that we are observing. 

Through observing the magnitudes on different days.

Could we confirm the expansion rate of the universe through our observations of supernovae? 

\vspace{-3ex}
\section{Observations} 
\vspace{-2ex}

asdasd

\vspace{-3ex}
\section{Analysis}
\vspace{-2ex}

asdasd

\vspace{-3ex}
\section{Discussion}
\vspace{-2ex}

asdasd

\vspace{-5ex}
\section{Conclusions}
\vspace{-2ex}

sdfsdf

\vspace{-5ex}
\section{Acknowledgements}
\vspace{-2ex}

sdfsadf

\begin{thebibliography}{5}
\bibitem{mathmethods}
	K. F. Riley, M. P. Hobson, and S. J. Bence.
	\textit{Mathematical Methods for Physics and Engineering}.
	Cambridge University Press, Cambridge, UK, 2010.	
\end{thebibliography}
\clearpage

\vfill
\twocolumngrid
\vspace{-3ex}
\section*{Appendix A - Observation Logs}
\vspace{-2ex}



\clearpage
\end{document}