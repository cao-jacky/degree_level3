\documentclass[twocolumn]{revtex4}
\usepackage{graphics,graphicx,epsfig,amsmath,multirow,gensymb,commath,textcomp,natbib,blindtext,mhchem,tabularx,array,makecell,threeparttable,amssymb,relsize,csquotes}
\usepackage{listings}
\newcommand{\squeezeup}{\vspace{-2.5mm}}
\usepackage[a4paper, left=1.85cm, right=1.85cm, top=1.85cm, bottom=1.85cm]{geometry}   

\def\bibsection{\section*{\refname}} 
\renewcommand{\thesubsection}{\alph{subsection}}

\renewcommand\theadalign{bc}
\renewcommand\theadfont{\bfseries}
\renewcommand\theadgape{\Gape[4pt]}
\renewcommand\cellgape{\Gape[4pt]}

\begin{document}

\textheight=26.385cm
%Change textheight as the last resort...

\title{Constraining the geometry of the Universe using Type Ia supernovae}
 
\author{Jacky Cao, Group C3, Physics Problem Solving \\ Date of report: 28/02/2018}

\begin{abstract}              
Type Ia supernovae have the unique trait of being standard candles, their light curves can be used in cosmology to calculate and constrain cosmological parameters. In observing Type Ia supernovae and fitting model light curves to such data one can attempt to derive such values. We have monitored, collected, and analysed data for supernova explosions over a period of 34 days. A $16''$ and a $0.5$ m telescope situated in Durham and La Palma was used for this project. After calculating the magnitudes for a Type Ia (2017hhz) and Type Ia-91bg (2017hle) supernova object, we fitted template light curves with the Python program, \textit{SNooPy}. The quality of fit for the program's \texttt{fit()} function was deemed to be acceptable in accordance to the average reduced $\chi^2$ values calculated for the B and V photometric bands, $\chi^2_{\nu,B} \approx 1.38$ and $\chi^2_{V,\nu} \approx 2.95$ - a good fit requiring $\chi^2_{\nu} \approx1$. The distance modulus to the supernova 2017hhz was calculated by SNooPy to be $36.121\pm0.106$ mag, using this value we were able to compute $H_0=70\pm20$ km s$^{-1}$ Mpc$^{-1}$. However, the quoted error negates the meaning of $H_0$ as it is too large of an uncertainty. In improving the accuracy and uncertainties we suggest that more observations of the supernovae were required, and constraining values should be used for the parameters in SNooPy's templates.
\end{abstract}

\maketitle

\vspace{-3ex}
\section{Introduction} 
\vspace{-2ex}
\subsection{Supernovae}
\vspace{-2ex}

In cosmology, we can argue that one of the most important objects that we can observe are supernovae.  \cite{obs_uni}

\vspace{-3ex}
\subsection{Cosmological Parameters}
\vspace{-2ex}

sdfsdf

\vspace{-3ex}
\section{Results} 
\vspace{-2ex}

asdasd

\vspace{-3ex}
\section{Discussion}
\vspace{-2ex}

asdasd

\vspace{-5ex}
\section{Conclusions}
\vspace{-2ex}

sdfsdf

\vspace{-5ex}
\section{Acknowledgements}
\vspace{-2ex}

sdfsadf

\vspace{-3ex}
\bibliographystyle{abbrv}
\bibliography{supernova_cosmology}

\clearpage
\appendix

\vfill
\twocolumngrid
\vspace{-3ex}
\section{Observation Logs}
\vspace{-2ex}



\clearpage
\end{document}