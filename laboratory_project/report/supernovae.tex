\documentclass[twocolumn]{revtex4}
\usepackage{graphics,graphicx,epsfig,amsmath,multirow,gensymb,commath,textcomp,natbib,blindtext,mhchem}
\usepackage[normalem]{ulem}
\newcommand{\squeezeup}{\vspace{-2.5mm}}

\def\bibsection{\section*{\refname}} 
\renewcommand{\thesubsection}{\alph{subsection}}

\begin{document}

\textheight=26.385cm
%Change textheight as the last resort...

\title{Studying the evolution of supernova magnitudes}
 
\author{Jacky Cao, AstroLabs, Lab Partner: Duncan Middlemiss \\ Dates of experiment: 19/10/2017 to 17/03/2017, Date of report: 19/03/2017}

\begin{abstract}              
We have measured the magnitude of supernova explosions over an extended period of 47 days using $0.5$ m and $?.?$ m telescopes situated in Durham and La Palma. We have plotted several light curves and identified Type Ia, Type II, and ?? supernovae. Our fittings have had $\chi^2$ analysis has performed, and it has produced values of ?, ?, ?. We expect our biggest source of uncertainty arose due to the conditions and data analysis that we performed. Using a Type Ia supernova of brightness $00.00$ mag, we have managed to produce a value for Hubble's Constant, $H_0 = 74$ kms$^{-1}$ Mpc$^{-1}$. We attempted to calculate Einstein's coefficient, $\Lambda$, but this was unsuccesful due to redsjhift of something.
\end{abstract}

\maketitle

\vspace{-3ex}
\section{Introduction} 
\vspace{-2ex}
\subsection{Supernovae}
\vspace{-2ex}

One of the potential fates for a massive star is an extremely violent and luminous explosion. The likes of which occur when there are no possible equilibrium configurations for a star to exist in after it has ran out of nuclear fuel to burn. The luminosity of such an explosion when it reaches it's peak can be as bright as a small galaxy \cite{longair}.

Observing this event and measuring it's magnitude over a period allows us to plot {\em light curves\em}, where magnitude can be displayed as a function of time. If we were to then collect sufficiently enough data from multiple supernovae, groups would begin to form. This is one way in which we can classify the different types of supernova explosions. 

In general there are two basic classifications of supernova, Type I and Type II. The main distinction between both of them is that Type I has an absence of the Balmer series of hydrogen in their optical spectrum at maximum light, and Type II does have this hydrogen feature in it's spectrum \cite{mod_ast, longair}. We can further split the two groups into more sub types, as can be seen in Table \ref{sn_classes}. These arise due to differences in spectral features and in their light curves \cite{longair}.

\begin{table}[h!]
\centering
\begin{tabular}{c@{\hskip 20pt}c} 
 \hline
 \textbf{Type} & \textbf{Characteristic} \\ 
 Type Ia		& Si II line at $616.0$ nm \\
 			& \em Type Ib and Type Ic also exist \em \\
 Type IIP 		& Reaches a plateau in it's light curve \\
 Type IIL		& Displays a linear decrease in it's light curve \\
 			& \em Type IIn and Type IIb also exist \em \\
 \hline
\end{tabular}
\caption{Some of the subclassifications of supernova \cite{longair}.}
\label{sn_classes}
\end{table}

In exploring the types of supernova explosions (SNe) we find that each have had very different lead ups to end up where they are. 

\vspace{-3ex}
\subsubsection{Type I Supernovae}
\vspace{-2ex}
With the case of Type I supernovae, the general consensus is that presupernova, a white dwarf within a binary system accretes matter from a donor star. The white dwarf then increases in mass until it reaches a critical point \cite{posn, longair}. This limit is thought to be the Chandrasekhar mass, $1.4 M_{\odot}$, after accreting this much matter, the object will be gravitationally unstable \cite{longair}. As the material is continually compressed and heated to a temperature of $T \geq 10^9$ K, the fusion of carbon and oxygen begins in the core which releases thermonuclear energy. This energy that is produced disrupts the star at high velocity and prevents a collapse into a neutron star, thus we have a supernova explosion \cite{posn}.

One of the subgroups of Type I which are especially important for cosmology are Type Ia SNe. When plotting their magnitudes against time passed we see that their light curves are generally homogeneous. These objects are the most luminous supernovae known, their absolute magnitudes in the B band are typically $M_B = -19.5 \pm 0.1$ \cite{posn}. Using light curve relationships and models we can determine precisely the magnitudes of very distant SNe, this has allowed astronomers to determine the redshift-distance relation for redshifts high redshift objects ($z>1$). Then, with these calculated values it is then possible to estimate the cosmological parameters $\Omega_0$ and $\Omega_\Lambda$. In doing so, it has been found that using Type Ia SNe has produced agreement with the literature value. so and so (cite paper) have done this and produced values of. 

\vspace{-3ex}
\subsubsection{Type II Supernovae}
\vspace{-2ex}

For a star to end it's life as a Type II supernova explosion, it's core must collapse to produce the enormous amounts of energy that we associate with a SNe. For stars more massive than $8 M_{\odot}$, their post-main-sequence evolution relies on the continual burning of carbon, oxygen, and silicon burning. As this process continues to heat up the core to larger temperatures, photodisintegration will start to occur. Photons have enough energy to breakup heavier elements such as iron (\ce{^{56}_{26}Fe}) and helium (\ce{^{4}_{2}He}), this process is endothermic so energy is removed from the gas that would have created the pressure required to support the core of the star \cite{mod_ast}. 

The free electrons which had been supporting the star through degeneracy pressure are now captured by heavy nuclei and by the protons that were produced through photodisintegration. Through the photodisintegration of iron, and combined with electron capture by other elements, most of the core's support disappears and then the core begins to collapse extremely rapidly. 

\vspace{-3ex}
\subsection{Supernova Discovery}
\vspace{-2ex}

All sky camera? [find source for this]

\vspace{-3ex}
\subsection{Application to Cosmology}
\vspace{-2ex}

Type Ia supernovae are useful to us as they can be classed as standard candles [cite]. With this we can use them to calculate a value for Hubble's Constant, $H_0$, and thus calculate the expansion rate of the Universe.

\vspace{-3ex}
\subsection{Project Aims}
\vspace{-2ex}

Studying the evolution of the magnitudes of different supernovae will allow us to produce a light curve which can be fitted with known models. Through this we can then discover the type of supernova that we are observing. 

The main focus in experimentation is attempting to collect the data which would form our light curves. 

Could we confirm the expansion rate of the universe through our observations of supernovae? 

\vspace{-3ex}
\section{Observations} 
\vspace{-2ex}
\subsection{Data Collection}
\vspace{-2ex}

sdafsdf

\vspace{-3ex}
\subsection{Observations Made}
\vspace{-2ex}

asdasd

\vspace{-3ex}
\subsection{Data Analysis}
\vspace{-2ex}

sdfsdf

\vspace{-3ex}
\subsection{Data Uncertainties}
\vspace{-2ex}

asdasdasd

\vspace{-3ex}
\subsection{Final Data}
\vspace{-2ex}

asdasd

\vspace{-3ex}
\section{Analysis}
\vspace{-2ex}
\subsection{Supernovae Models}
\vspace{-2ex}

The fitting of the templates/models to our observations.

\vspace{-3ex}
\subsection{Resuluts}
\vspace{-2ex}

sdfsdfsa

\vspace{-3ex}
\section{Discussion}
\vspace{-2ex}

It is highly unlikely that we will see supernova remnants for the supernova that we have been studying, in the shape of the Crab Nebula. Much longer time periods would be required for this. 

[neutrino studying]?

asdasd

\vspace{-5ex}
\section{Conclusions}
\vspace{-2ex}

sdfsdfasdas

\vspace{-5ex}
\section*{Acknowledgements}
\vspace{-2ex}

I would like to thank Carl Sagan for producing his Cosmos series and inspiring a generation to become astronomers.


\bibliographystyle{abbrv}
\bibliography{supernovae}
\clearpage




\vfill
\twocolumngrid
\vspace{-3ex}
\section*{Appendix A - Observation Logs}
\vspace{-2ex}



\clearpage
\end{document}